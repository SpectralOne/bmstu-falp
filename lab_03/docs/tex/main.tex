\chapter*{Лабораторная работа №1}

\section*{Задание 1}

Составить диаграмму вычисления следующих выражений:

\begin{lstinputlisting}[
	caption={Задание 1},
	label={lst:t1},
	style={lsp},
	linerange={3-8},
	]{../src/main.lsp}
\end{lstinputlisting}

\begin{lstlisting}[style={scheme}]
1. (equal 3 (abs -3))
     3 вычисляется как 3
     (abs -3)
       -3 вычисляется как -3
       abs применяется к -3
       3
     equal применяется к 3 и 3
     T
\end{lstlisting}

\begin{lstlisting}[style={scheme}]
2. (equal (+ 1 2) 3)
     (+ 1 2)
       1 вычисляется как 1
       2 вычисляется как 2
       + применяется к 1 и 2
       3
     3 вычисляется как 3
     equal применяется к 3 и 3
     T
\end{lstlisting}

\begin{lstlisting}[style={scheme}]
3. (equal (* 4 7) 21)
     (* 4 7)
       4 вычисляется как 4
       7 вычисляется как 7
       * применяется к 4 и 7
       28
     21 вычисляется как 21
     equal применяется к 28 и 21
     Nil
\end{lstlisting}

\begin{lstlisting}[style={scheme}]
4. (equal (* 2 3) (+ 7 2))
     (* 2 3)
       2 вычисляется как 2
       3 вычисляется как 3
       * применяется к 2 и 3
       6
     (+ 7 2)
       7 вычисляется как 7
       2 вычисляется как 2
       + применяется к 7 и 2
       9
     equal применяется к 6 и 9
     Nil
\end{lstlisting}

\begin{lstlisting}[style={scheme}]
5. (equal (- 7 3) (* 3 2))
     (- 7 3)
       7 вычисляется как 7
       3 вычисляется как 3
       - применяется к 7 и 3
       4
     (* 3 2)
       3 вычисляется как 3
       2 вычисляется как 2
       * применяется к 3 и 2
       6
     equal применяется к 4 и 6
     Nil
\end{lstlisting}

\begin{lstlisting}[style={scheme}]
6. (equal (abs (- 2 4)) 3)
     (abs (- 2 4))
       (- 2 4)
         2 вычисляется как 2
         4 вычисляется как 4
         - применяется к 2 и 4
         -2
       abs применяется к -2
       2
     3 вычисляется как 3
     equal применяется к 2 и 3
     Nil
\end{lstlisting}

\section*{Задание 2}

Написать функцию, вычисляющую гипотенузу прямоугольного треугольника по заданным катетам и составить диаграмму её вычисления.

\begin{lstinputlisting}[
	caption={Задание 2},
	label={lst:t2},
	style={lsp},
	linerange={11-16},
	]{../src/main.lsp}
\end{lstinputlisting}

\begin{lstlisting}[style={scheme}]
(a-square (* a a))
  (* a a)
    a вычисляется как a
    a вычисляется как a
    * применяется к a и a
    $a^2$
  привязка $a^2$ к a-square
(b-square (* b b))
  (* b b)
    b вычисляется как b
    b вычисляется как b
    * применяется к b и b
    $b^2$
  привязка $b^2$ к b-square
(square-sum (+ a-square b-square)))
  (+ a-square b-square)
    + применяется к a-square и b-square
    $a^2 + b^2$
  привязка $a^2 + b^2$ к square-sum
(sqrt square-sum)
  sqrt применяется к square-sum
  $\sqrt{a^2+b^2}$
\end{lstlisting}

\section*{Задание 3}

Написать функцию, вычисляющую объём параллелепипеда по 3-м его сторонам и составить диаграмму её вычисления.

\begin{lstinputlisting}[
	caption={Задание 3},
	label={lst:t3},
	style={lsp},
	linerange={18-20},
	]{../src/main.lsp}
\end{lstinputlisting}

\begin{lstlisting}[style={scheme}]
(* a b h)
  a вычисляется как a
  b вычисляется как b
  h вычисляется как h
  * приминяется к a, b, h
  $a\cdot b\cdot h$
\end{lstlisting}

\section*{Задание 4}

Каковы результаты вычисления следующих выражений?

\begin{lstinputlisting}[
	caption={Задание 4},
	label={lst:t4},
	style={lsp},
	linerange={24-31},
	]{../src/main.lsp}
\end{lstinputlisting}

\section*{Задание 5}

Написать функцию {\texttt{longer-than}} от двух списков-аргументов, которая возвращает {\texttt{T}}, если первый аргумент имеет большую длину.

\begin{lstinputlisting}[
	caption={Задание 5},
	label={lst:t5},
	style={lsp},
	linerange={34-38},
	]{../src/main.lsp}
\end{lstinputlisting}

\section*{Задание 6}

Каковы результаты вычисления следующих выражений?

\begin{lstinputlisting}[
	caption={Задание 6},
	label={lst:t6},
	style={lsp},
	linerange={42-48},
	]{../src/main.lsp}
\end{lstinputlisting}

\section*{Задание 7}

Дана функция 

\begin{lstinputlisting}[
	caption={mystery},
	label={lst:func},
	style={lsp},
	linerange={50-51},
	]{../src/main.lsp}
\end{lstinputlisting}

Каковы результаты вычисления следующих выражений?

\begin{lstinputlisting}[
	caption={Задание 7},
	label={lst:t7},
	style={lsp},
	linerange={53-56},
	]{../src/main.lsp}
\end{lstinputlisting}

\section*{Задание 8}

Написать функцию, которая переводит температуру в системе Фаренгейта в температуру по Цельсию. $f = \cfrac{9}{5}\cdot c + 32$

\begin{lstinputlisting}[
	caption={Задание 8},
	label={lst:t8},
	style={lsp},
	linerange={59-61},
	]{../src/main.lsp}
\end{lstinputlisting}

\noindentКак бы назывался роман Р. Бредбери <<+451 по Фаренгейту>> в системе по Цельсию?\\
-- <<232.78 по Цельсию>>

\section*{Задание 9}

Что получится при вычислении каждого из выражений?

\begin{lstinputlisting}[
	caption={Задание 9},
	label={lst:t9},
	style={lsp},
	linerange={65-71},
	]{../src/main.lsp}
\end{lstinputlisting}

\section*{Дополнительное задание 1}

Написать функцию, вычисляющую катет по заданной гипотенузе и другому катету прямоугольного треугольника и составить диаграмму её вычисления. 

\begin{lstinputlisting}[
	caption={Дополнительное задание 1},
	label={lst:e1},
	style={lsp},
	linerange={74-79},
	]{../src/main.lsp}
\end{lstinputlisting}

\begin{lstlisting}[style={scheme}]
(hyp-square (* hyp hyp))
  (* hyp hyp)
    hyp вычисляется как hyp
    hyp вычисляется как hyp
    * применяется к hyp и hyp
    $hyp^2$
  привязка $hyp^2$ к hyp-square
(side-square (* side side))
  (* side side)
    side вычисляется как side
    side вычисляется как side
    * применяется к side и side
    $side^2$
  привязка $side^2$ к side-square
(square-sub (- hyp-square side-square)))
  (- hyp-square side-square)
    - применяется к hyp-square и side-square
    $hyp^2 - side^2$
  привязка $hyp^2 - side^2$ к square-sub
(sqrt square-sub)
  sqrt применяется к square-sub
  $\sqrt{hyp^2-side^2}$
\end{lstlisting}

\section*{Контрольные вопросы}

\subsection*{1. Базис языка {\texttt{Lisp}}}

Базис языка представлен:
\begin{itemize}
	\item структурами, атомами;
	\item функциями:\\
	{\texttt{atom, eq, cons, car, cdr,}}\\
	{\texttt{cond, quote, lambda, eval, label}}.
\end{itemize}

\subsection*{2. Классификация функций языка {\texttt{Lisp}}}

Функции в языке {\texttt{Lisp}}:
\begin{itemize}
	\item чистые (с фиксированным количеством аргументов) -- математические функции;
	\item рекурсивные функции;
	\item специальные функции -- формы (принимают произвольное количество аргументов или по разному обрабатывают аргументы);
	\item псевдофункции (создающие <<эффект>> - отображающие на экране процесс обработки данных и т. п.);
	\item функции с вариативными значениями, выбирающие одно значение;
	\item функции высших порядков -- функционалы (используются для построения синтаксически управляемых программ).
\end{itemize}

\subsection*{3. Синтаксис элементов языка и их представление в памяти}


\noindent{\texttt{Точечные пары ::= (<атом>, <атом>) |}}\\

{\texttt{(<атом>, <точечная пара>) |}}\\

{\texttt{(<точечная пара>, <атом>) |}}\\

{\texttt{(<точечная пара>, <точечная пара>)}}\\


\noindent{\texttt{Список ::= <пустой список> | <непустой список>)}}, где\\

{\texttt{<пустой список> ::= () | Nil}},\\

{\texttt{<непустой список> ::= (<первый элемент>, <хвост>) }},\\

{\texttt{<первый элемент> ::= (S-выражение)}},\\

{\texttt{<хвост> ::= <список>}}\\


\noindent Список -- частный случай {\texttt{S-выражения}}.\\

Синтаксически любая структура (точечная пара или список) заключается в {\texttt{()}}:\\

{\texttt{(A . B)}} -- точечная пара\\

{\texttt{(A)}} -- список из одного элемента\\

Пустой список изображается как {\texttt{Nil}} или {\texttt{()}}\\

Непустой список может быть изображён: {\texttt{(A. (B . (C ())))}} или {\texttt{(A B C)}}\\

Элементы списка могут являться списками: {\texttt{((A) (B) (C))}}\\

Любая непустая структура {\texttt{Lisp}} в памяти представлена списковой ячейкой, хранящей два указателя: на голову (первый элемент) и хвост (всё остальное).


\subsection*{4. Функции {\texttt{car, cdr}}}

Являются базовыми функциями доступа к данным. {\texttt{car}} принимает точечную пару или список в качестве аргумента и возвращает первый элемент или {\texttt{Nil}}, {\texttt{cdr}} -- возвращает все элементы, кроме первого или {\texttt{Nil}}.

\subsection*{5. Функции {\texttt{list, cons}}}

Являются функциями создания списков ({\texttt{cons}} -- базовая, {\texttt{list}} -- нет). {\texttt{cons}} создаёт списочную ячейку и устанавливает два указателя на аргументы. {\texttt{list}} принимает переменное число аргументов и возвращает список, элементами которого являются аргументы, переданные в функцию.