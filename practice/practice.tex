% !TeX TXS-program:compile = txs:///lualatex/[--shell-escape]

\include{preamble.inc.tex}

\begin{document}
	
\begin{titlepage}
	\noindent\begin{minipage}{0.05\textwidth}
		\includegraphics[scale=0.3]{inc/bmstu.png}
	\end{minipage}
	\hfill
	\begin{minipage}{0.85\textwidth}\raggedleft
		\begin{center}
			\fontsize{12pt}{0.3\baselineskip}\selectfont \textbf{Министерство науки и высшего образования Российской Федерации \\ Федеральное государственное бюджетное образовательное учреждение \\ высшего образования \\ <<Московский государственный технический университет \\ имени Н.Э. Баумана \\ (национальный исследовательский университет)>> \\ (МГТУ им. Н.Э. Баумана)}
		\end{center}
	\end{minipage}

	\begin{center}
		\fontsize{12pt}{0.1\baselineskip}\selectfont
		\noindent\makebox[\linewidth]{\rule{\textwidth}{4pt}} \makebox[\linewidth]{\rule{\textwidth}{1pt}}
	\end{center}

	\begin{flushleft}
		\fontsize{12pt}{0.8\baselineskip}\selectfont 
		
		ФАКУЛЬТЕТ \uline{<<\textbf{Информатика и системы управления}>> \hfill}

		КАФЕДРА \uline{\mbox{\hspace{4mm}} <<\textbf{Программное обеспечение ЭВМ и информационные технологии}>> \hfill}
	\end{flushleft}

	\vfill

	\begin{center}
		\fontsize{20pt}{\baselineskip}\selectfont

		\uline{\textbf{ОТЧЁТ ПО ПРОИЗВОДСТВЕННОЙ ПРАКТИКЕ}}
	\end{center}
	
	\vfill
	
	\begin{flushleft}
		\fontsize{12pt}{0.7\baselineskip}\selectfont

		Студент \uline{\mbox{\hspace{44mm}} Богаченко Артём Евгеньевич \hfill}
		
		Группа \uline{\mbox{\hspace{64mm}} ИУ7-65Б \hfill}
		
		Тип практики \uline{\mbox{\hspace{44mm}} Производственная \hfill}
		
		Название предприятия \uline{\mbox{\hspace{26mm}} ООО~<<Эррайвал РУС>> \hfill}
	\end{flushleft}	

	\vfill

	\begin{table}[h!]
		\fontsize{12pt}{0.7\baselineskip}\selectfont

		\begin{signstabular}[0.55]{p{7.25cm} >{\centering\arraybackslash}p{4cm} >{\centering\arraybackslash}p{4cm}}
		Студент & \uline{\mbox{\hspace*{4cm}}} & \uline{\hfill \textbf{Богаченко А. Е.} \hfill} \\
		& \scriptsize \textit{подпись, дата} & \scriptsize \textit{фамилия, и.о.}
		\end{signstabular}
	
		\vspace{\baselineskip}

		\begin{signstabular}[0.55]{p{7.25cm} >{\centering\arraybackslash}p{4cm} >{\centering\arraybackslash}p{4cm}}
			Руководитель практики & \uline{\mbox{\hspace*{4cm}}} & \uline{\hfill \textbf{Толпинская Н. Б.} \hfill} \\
			\mbox{\hspace*{1cm}} \scriptsize (от университета) & \scriptsize \textit{подпись, дата} & \scriptsize \textit{фамилия, и.о.}
		\end{signstabular}

		\vspace{\baselineskip}
		
		\begin{signstabular}[0.55]{p{7.25cm} >{\centering\arraybackslash}p{4cm} >{\centering\arraybackslash}p{4cm}}
			Руководитель практики & \uline{\mbox{\hspace*{4cm}}} & \uline{\hfill \textbf{Холодная Т. Ю.} \hfill} \\
			\mbox{\hspace*{1cm}} \scriptsize (от предприятия) & \scriptsize \textit{подпись, дата} & \scriptsize \textit{фамилия, и.о.}
		\end{signstabular}
	
		\vspace{\baselineskip}
		
		\begin{signstabular}[0.55]{p{7.25cm} >{\centering\arraybackslash}p{4cm} >{\centering\arraybackslash}p{4cm}}
			Оценка~~\uline{\hfill}
		\end{signstabular} 
	 
	\end{table}

	\vfill

	\begin{center}
		\normalsize \textit{\the\year~г.}
	\end{center}
\end{titlepage}
\normalsize

\pagenumbering{arabic}
\setcounter{page}{2}

\tableofcontents
\normalsize

\pagebreak

\specsection{ВВЕДЕНИЕ}

Цель производственной практики -- осуществить выборку данных на основе критерия для последующей передачи в сервис по созданию отчётов, привести данные в соответствии со внешними требованиями сервиса\cite{microservice}. 

Задачи практики:
\begin{enumerate}
	\item утвердить техническое задание с заказчиком;
	\item провести анализ используемых предприятием технологий;
	\item ознакомиться с правилами создания запроса для функционально-\\* логической базы данных;
	\item изучить инструменты для обработки и конвертации данных;
	\item изучить способы сериализации\cite{serde} данных.
\end{enumerate}

\clearpage

\section{Характеристика предприятия}

ООО «Эррайвал РУС»\cite{arrival} было основано в 2015 году. Основная сфера деятельности – производство электрических транспортных средств. Головной офис расположен в Лондоне, акции находятся в свободном торговом доступе на международных биржах.

Кроме производства транспортных средств компания занимается разработкой собственного программного обеспечения для работы и обслуживания производимых продуктов, обучения сотрудников и ведения промо компаний. Особое внимание уделяется качеству, надёжности и удобству используемых продуктов -- от надёжности и качества напрямую зависит безопасность людей, использующих продукты компании.

Ключевой особенностью является собственное производство всех комплектующих, использование микро-фабрик и сборка транспортных средств без участия человека.

\clearpage

\section{Задание для производственной практики}

Требуется по запросу сервиса по созданию отчётов получить и подготовить выборку из базы данных используя переданный сервисом критерий. Составляемый в базу данных запрос должен следовать принципу уменьшения итоговой выборки данных каждым последующим правилом. Подготовленную выборку нужно передать сервису создания отчётов в сериализованном транспортном \cite{DTO} формате.

\clearpage

\section{Описание выполнения задания}

\subsection{Анализ используемых технологий}

\subsubsection{Основное приложение}

В качестве языка программирования основного приложения используется функциональный язык \texttt{Clojure} \cite{clojure}, являющийся подмножеством языка \texttt{LISP} \cite{lisp} и поддерживающий библиотеки JVM \cite{jvm}, подходящий для обработки больших объёмов данных и построения высоконагруженных систем.

\subsubsection{Сервис по созданию отчётов}

Сервис по созданию отчётов в качестве основного языка программирования использует \texttt{Kotlin}\cite{kotlin}. Для корректного преобразования объектов используется библиотека \texttt{Jackson}\cite{jackson}.

Для передачи данных от приложения сервису была написана библиотека, определяющая транспортные типы данных. Данные передаются в виде сериализованных \texttt{JSON}-объектов\cite{JSON}. Сервис при передаче критерия отчёта приложению также использует \texttt{JSON}-объект.

\subsubsection{База данных приложения}

В качестве системы управления базой данных\cite{subd} (СУБД) используется \\* \texttt{Datomic}\cite{datomic} -- распределённая база данных класса NoSQL\cite{nosql}, реализующая язык логических операций \texttt{Datalog} \cite{datalog}. В качестве сервиса хранения используется база данных\cite{database} \texttt{PostgreSQL}\cite{postgresql}.

Особенностью рассматриваемой СУБД является связь каждой транзакции с текущим временем, база данных хранит историю всех операций, что позволяет получить её состояние в любой момент времени, такой подход позволяет реализовать режим просмотра истории изменения данных, являющийся аналогом журналирования.

\texttt{Datomic} предоставляет транзакции, соответствующие принципам \\* \texttt{ACID}\cite{acid} и состоит из механизма изменения состояния базы данных -- транзактора, сервиса хранилища данных и пользовательского сервиса, который отсылает данные на транзактор. Оптимальная скорость работы достигается за счёт зарезервированных инструкций и консистентности данных, встроенного в транзактор механизма кэширования запросов как на чтение, так и на запись, и отсутствия планировщика транзакций. Унификация данных структурой вида <<ключ-значение>> позволяет достичь согласованности, в качестве ключа может быть как зарезервированное слово, так и атрибут сущности или хранимая процедура.

Главной особенностью является возможность пользователя полностью влиять на системные настройки и отсутствие явных правил формирования запроса. Неумелое обращение с данной СУБД может привести к результатам на порядок хуже, чем в традиционных реализациях, например, отсутствие планировщика транзакций может значительно замедлить выполнение составного запроса.

\subsection{Создание запроса}

База данных представлена в виде фактов и отношений\cite{datomic-facts}, каждая запись имеет следующую структуру: \texttt{[}сущность атрибут значение транзакция \texttt{'\_]}, запрос представлен хеш-картой\cite{hash-map} вида <<ключ-значение>>.

Структура запроса состоит из трёх ключей:
\begin{itemize}
	\item \texttt{:find} -- содержит свободный набор переменных, которые будут использованы для итогового формирования выборки;
	\item \texttt{:in} -- содержит источники данных, которые будут использованы в запросе;
	\item \texttt{:where} -- содержит набор правил, которые записывают значения атрибутов в свободные переменные и используют источники данных для постановки логического вопроса.
\end{itemize}

Например, для того, чтобы в базе данных музыкантов и их музыкальных релизов найти все релизы по имени музыканта, можно использовать запрос, представленный в листинге \ref{code:query}. \\

\begin{lstlisting}[label=code:query, caption={Пример запроса}]
(d/q {:find ?release-name
      :in $ ?artist-name
      :where [?artist :artist/name ?artist-name]
             [?release :release/artists ?artist]
             [?release :release/name ?release-name]}
      db "John Lenon")
      
=> #{["Power to the People"] 
     ["Unfinished Music No. 2: Life With the Lions"] 
     ["Live Peace in Toronto 1969"] 
     ["Live Jam"]
     ...}
\end{lstlisting}

Для формирования запроса используется хеш-карта, каждый ключ запроса будет формироваться на основе наличия или отсутствия параметров критерия. Обновление параметров запроса осуществляется с помощью оператора условного перехода \texttt{cond} вместе с потоковым макросом\cite{threading-macro} \texttt{->}. Для обновления значения ключей используются стандартные операторы \texttt{update} и \texttt{conj}. Пример формирования запроса представлен в листинге \ref{code:query-build}. Результатом запроса является коллекция сущностей, отфильтрованная с помощью критерия.

\begin{lstlisting}[label=code:query-build, caption={Создание запроса}]
(cond-> '{:find [?x]
          :in [$]
          :where []}
  report-param-1
  (-> (update :in conj '[?report-param-1...])
      (update :where
              conj
              '[?rp-1 ?attr ?report-param-1]))
  ...
\end{lstlisting}

\subsection{Обработка и конвертация данных}

Чтобы сконвертировать выборку в коллекцию транспортных объектов необходимо предварительно получить все необходимые значения полей. Каждая сущность представляет собой хеш-карту, достаточно извлечь необходимые значения атрибутов. Пример деструктуризации\cite{destructuring} сущности представлен в листинге \ref{code:destructure}.

\begin{lstlisting}[label=code:destructure, caption={Деструктуризация сущности}]
(defn- rasterize
  [{:entity/keys [key-1 key-2 key-3]}]
  (assoc-some
    {:key-1 key-1
     :key-2 key-2
     :key-3 key-3}))
 
(->> query-result
     (map rasterize))
\end{lstlisting}

Для конвертации был написан макрос преобразования типов полей. Конвертация происходит согласно заранее определённому типу транспортного объекта. Для каждого типа отчёта, создаётся своя функция для конвертации. Данный подход позволяет расширять список поддерживаемых типов отчётов.

\clearpage
\begin{lstlisting}[label=code:macro, caption={Макрос конвертации в транспортный объект}]
(defmacro def-converter
  [report-type]
  (let [dto-class (symbol (csk/->PascalCaseString report-type))
        type-info (clojure.reflect/reflect (resolve dto-class))
        members (:members type-info)
        setters (filter #(string/starts-with? (:name %) "set")) members)
        fn-name (symbol (str "make-" report-type "-record"))]
    `(defn ~(with-meta fn-name {:report-type report-type})
       [~'item]
       (let [~'record (new ~dto-class)]
         ~@(map (fn [setter]
                  (let [setter-name (str (:name setter))
                        item-key (csk/->kebab-case-keyword
                                  (subs setter-name (count "set")))
                        setter-fn (symbol (str "." setter-name))
                        [field-type] (:parameter-types setter)
                        converter-fn (or (get field->converter item-key)
                                         (case field-type
                                           java.lang.Long 'long
                                           java.lang.Float 'float
                                           dto.type-1 'coverter-type-1
                                           nil))]
                    (if (= field-type 'java.util.List)
                      (let [converter-fn (get field->converter item-key)]
                        `(let [~'l (ArrayList.)]
                           (doseq [~'v (~item-key ~'item)]
                             (.add ~'l ~(if converter-fn
                                          `(~converter-fn ~'v)
                                          'v)))
                           (~setter-fn ~'record ~'l)))
                      `(~setter-fn ~'record 
                                   ~(if converter-fn
                                        `(~converter-fn (~item-key ~'item))
                                        `(~item-key ~'item))))))
                setters)
         ~'record))))
\end{lstlisting}

\subsection{Сериализация объектов}

Для передачи данных в сервис была написана функция сериализации в \texttt{JSON}-объект. Исходный код функции представлен в листинге \ref{code:serialize}.

\begin{lstlisting}[label=code:serialize, caption={Сериализация объектов}]
(fn [stream]
  (try
    (with-open [writer (io/writer stream)]
      (let [^ObjectMapper object-mapper (make-object-mapper)]
        (let [report-info (doto (ReportRequestInfo.) ...)
              json-string (.writeValueAsString object-mapper report-info)]
          (.write writer json-string)
          (.write writer "\r\n"))

        (doseq [item items]
          (let [json-string (.writeValueAsString object-mapper
                                                 (make-record item))]
            (.write writer json-string)
            (.write writer "\r\n")))

        (.write writer ".")
        (.write writer "\r\n")))
    (catch Throwable t
      (log/error t "Unable to transfer report data")
      (throw t))))
\end{lstlisting}

\clearpage


\specsection{ЗАКЛЮЧЕНИЕ}

В рамках производственной практики был реализован функционал получения выборки по заданному критерию, обработки, конвертации и сериализации данных.

Так же были выполнены следующие задачи:
\begin{enumerate}
	\item утверждено техническое задание;
	\item проведён анализ используемых предприятием технологий;
	\item ознакомился с правилами создания запроса для функционально-\\* логической базы данных;
	\item изучил инструменты для обработки и конвертации данных;
	\item изучил способы сериализации данных.
\end{enumerate}


\clearpage

\specsection{СПИСОК ИСПОЛЬЗОВАННЫХ ИСТОЧНИКОВ}

\begingroup
\renewcommand{\section}[2]{}
\bibliographystyle{utf8gost705u}
\bibliography{bibliography}   
\endgroup

\end{document}
