\section*{1. Написать функцию, которая по своему аргументу-списку \texttt{lst} определяет, является ли он полиндромом (то есть равны ли \texttt{lst} и \texttt{(reverse lst)})}

\begin{lstinputlisting}[
	caption={Задание 1},
	label={lst:t1},
	style={lsp},
	linerange={1-2},
	]{../src/main.lsp}
\end{lstinputlisting}

\section*{2. Написать предикат \texttt{set-equal}, который возвращает \texttt{t}, если два его множества-аргумента содержат одни и те же элементы, порядок которых не имеет значения}

\begin{lstinputlisting}[
	caption={Задание 2},
	label={lst:t2},
	style={lsp},
	linerange={4-5},
	]{../src/main.lsp}
\end{lstinputlisting}

\section*{3. Напишите необходимые функции, которые обрабатывают таблицу из точечных пар: \texttt{(страна . столица)}, и возвращают по стране столицу, а по столице --- страну}

\clearpage

\begin{lstinputlisting}[
	caption={Задание 3},
	label={lst:t3},
	style={lsp},
	linerange={7-13},
	]{../src/main.lsp}
\end{lstinputlisting}

\section*{4. Напишите функцию \texttt{swap-first-last}, которая переставляет в списке аргументе первый и последний элементы}

\begin{lstinputlisting}[
	caption={Задание 4},
	label={lst:t4},
	style={lsp},
	linerange={15-25},
	]{../src/main.lsp}
\end{lstinputlisting}

\section*{5. Напишите функцию \texttt{swap-two}, которая переставляет в списке-аргументе два указанных своими порядковыми номерами элемента в этом списке}

\clearpage

\begin{lstinputlisting}[
	caption={Задание 5},
	label={lst:t5},
	style={lsp},
	linerange={27-44},
	]{../src/main.lsp}
\end{lstinputlisting}

\section*{6. Напишите две функции, \texttt{swap-to-left} и \texttt{swap-to-right}, которые производят круговую перестановку в списке-аргументе влево и вправо, соответственно}

\begin{lstinputlisting}[
	caption={Задание 6},
	label={lst:t6},
	style={lsp},
	linerange={46-53},
	]{../src/main.lsp}
\end{lstinputlisting}

\section*{Контрольные вопросы}

\subsection*{1. Структуроразрушающие и не разрушающие структуру списка функции}

\subsubsection{Не разрушающие структуру списка функции}

\begin{itemize}
	\item \texttt{append} --- Объединяет списки. Создает копию для всех аргументов, кроме последнего;
	
	\item \texttt{reverse} --- Возвращает копию исходного списка, элементы которого переставлены в обратном порядке (работает только на верхнем уровне);
	
	\item \texttt{last} --- Возвращает последнюю списковую ячейку верхнего уровня;
	
	\item \texttt{nth} --- Возвращает указателя от n-ной списковой ячейки, нумерация с нуля;
	
	\item \texttt{nthcdr} --- Возвращает n-ого хвоста;
	
	\item \texttt{length} --- Возвращает длину списка (верхнего уровня);
	
	\item \texttt{remove} --- Данная функция удаляет элемент по значению (работает с копией), можно передать функцию сравнения через \texttt{:test};
	
	\item \texttt{rplaca} --- Переставляет \texttt{car}-указатель на 2 элемент-аргумент (\textit{S}-выражение);
	
	\item \texttt{rplacd} --- Переставляет \texttt{cdr}-указатель на 2 элемент-аргумент (\textit{S}-выражение);
	
	\item \texttt{subst} --- Заменяет все элементы списка, которые равны 2 переданному элементу-аргументу на другой 1 элемент-аргумент. \textit{По умолчанию для сравнения используется функция \texttt{eql}}.	
\end{itemize}

\subsubsection*{Структуроразрушающие функции}

Данные функции меняют сам объект-аргумент, невозможно вернуться к исходному списку. Чаще всего такие функции начинаются с префикса \texttt{n-}.

\begin{itemize}
	\item \texttt{nconc} --- Работает аналогично \texttt{append}, только не копирует свои аргументы, а разрушает структуру;
	
	\item \texttt{nreverse} --- Работает аналогично \texttt{reverse}, но не создает копии;
	
	\item \texttt{nsubst} --- Работае аналогично функции \texttt{nsubst}, но не создает копии;
	
\end{itemize}

\subsection*{2. Отличие в работе функций \texttt{cons}, \texttt{list}, \texttt{append}, \texttt{nconc} и в их результате}

Функция \texttt{cons} --- чисто математическая, конструирует списковую ячейку, которая может вовсе и не быть списком (будет списком только в том случае, если 2 аргументом передан список).

Примеры:
\begin{enumerate}
	\item \texttt{(cons 2 '(1 2))} --- \texttt{(2 1 2)} --- список;
	\item \texttt{(cons 2 3)} --- \texttt{(2 . 3)} --- не список.
\end{enumerate}

Функция \texttt{list} --- форма, принимает произвольное количество аргументов и конструирует из них список. Результат --- всегда список. При нуле аргументов возвращает пустой список.

\begin{enumerate}
	\item[] \texttt{(list 1 2 3)} --- \texttt{(1 2 3)};
	\item[] \texttt{(list 2 '(1 2))} --- \texttt{(2 (1 2))};
	\item[] \texttt{(list '(1 2) '(3 4))} --- \texttt{((1 2) (3 4))};
\end{enumerate}

Функция \texttt{append} --- форма, принимает на вход произвольное количество аргументов и для всех аргументов, кроме последнего, создает копию, ссылая при этом последний элемент каждого списка-аргумента на первый элемент следующего по порядку списка-аргумента (так как модифицируются все списки-аргументы, кроме последнего, копирование для последнего не делается в целях эффективности).

\begin{enumerate}
	\item[] \texttt{(append '(1 2) '(3 4))} --- \texttt{(1 2 3 4)};
	\item[] \texttt{(append '((1 2) (3 4)) '(5 6))} --- \texttt{((1 2) (3 4) 5 6)}.
\end{enumerate}
