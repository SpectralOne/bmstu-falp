\section*{1. Написать хвостовую рекурсивную функцию \texttt{my-reverse}, которая развернет верхний уровень своего списка-аргумента \texttt{lst}}

\begin{lstinputlisting}[
	caption={Задание 1},
	label={lst:t1},
	style={lsp},
	linerange={1-4},
	]{../src/main.lsp}
\end{lstinputlisting}

\section*{2. Написать функцию, которая возвращает первый элемент списка --- аргумента, который сам является непустым списком.}

\begin{lstinputlisting}[
	caption={Задание 2},
	label={lst:t2},
	style={lsp},
	linerange={6-9},
	]{../src/main.lsp}
\end{lstinputlisting}

\section*{3. Напишите функцию, select-between, которая из списка-аргумента, содержащего только числа, выбирает только те, которые расположены между двумя указанными границами аргументами и возвращает их в виде списка (упорядоченного по возрастанию списка чисел}

\begin{lstinputlisting}[
	caption={Задание 3},
	label={lst:t3},
	style={lsp},
	linerange={11-17},
	]{../src/main.lsp}
\end{lstinputlisting}

\section*{4. Написать рекурсивную версию (с именем \texttt{rec-add}) вычисления суммы чисел заданного списка}

\begin{lstinputlisting}[
	caption={Задание 4},
	label={lst:t4},
	style={lsp},
	linerange={19-22},
	]{../src/main.lsp}
\end{lstinputlisting}

\section*{5. Написать рекурсивную версию с именем recnth функции \texttt{nth}}

\begin{lstinputlisting}[
	caption={Задание 5},
	label={lst:t5},
	style={lsp},
	linerange={24-27},
	]{../src/main.lsp}
\end{lstinputlisting}

\section*{6. Написать рекурсивную функцию \texttt{allodd}, которая возвращает \texttt{T} когда все элементы списка нечетные.}

\begin{lstinputlisting}[
	caption={Задание 6},
	label={lst:t6},
	style={lsp},
	linerange={29-32},
	]{../src/main.lsp}
\end{lstinputlisting}

\section*{7. Используя cons-дополняемую рекурсию с одним тестом завершения,	написать функцию которая получает как аргумент список чисел, а возвращает список квадратов этих чисел в том же порядке}

\begin{lstinputlisting}[
	caption={Задание 7},
	label={lst:t7},
	style={lsp},
	linerange={34-36},
	]{../src/main.lsp}
\end{lstinputlisting}
