\textbf{Задание:} используя базу знаний, хранящую знания:
\begin{itemize}
	\item \textbf{<<Телефонный справочник>>:} Фамилия, №тел, Адрес - структура (Город, Улица, №дома, №кв),
	\item \textbf{<<Автомобили>>:} Фамилия\_владельца, Марка, Цвет, Стоимость, и др.,
	\item \textbf{<<Вкладчики банка>>:} Фамилия, Банк, счет, сумма, др.
\end{itemize}

Владелец может иметь несколько телефонов, автомобилей, вкладов (Факты). В разных городах есть однофамильцы, в одном городе --- фамилия уникальна.

Используя \textbf{Коньюктивное правило и простой вопрос}, обеспечить возможность поиска:

По Марке и Цвету автомобиля найти Фамилию, Город, Телефон и Банки, в которых владелец автомобиля имеет вклады. Лишей информации не находить и не передавать!!!

Владельцев может быть \textbf{несколько} (не более 3-х), \textbf{один} и \textbf{ни одного}.

\begin{enumerate}
	\item Для каждого из трёх вариантов \textbf{словесно подробно} описать порядок формирования ответа (в виде таблицы). При этом, указать – отметить моменты очередного запуска алгоритма унификации и полный результат его работы. Обосновать следующий шаг работы системы. Выписать унификаторы – подстановки. Указать моменты, причины и результат отката, если он есть.
	\item Для случая нескольких владельцев (2-х): приведите примеры (таблицы) работы системы \textbf{при разных порядках} следования в БЗ процедур, и знаний в них: (\textbf{<<Телефонный справочник>>, <<Автомобили>>, <<Вкладчики банков>>}, или: \textbf{<<Автомобили>>, <<Вкладчики банков>>, <<Телефонный справочник>>})) Сделайте вывод: Одинаковы ли: множество работ и объём работ в разных случаях?
	\item Оформите 2 таблицы, демонстрирующие \textbf{порядок работы алгоритма унификации} вопроса и подходящего заголовка правила (для двух случаев из пункта 2) и укажите результаты его работы: ответ и побочный эффект.
\end{enumerate}

\inputminted[
frame=single,
framesep=2mm,
baselinestretch=1.2,
bgcolor=white,
fontsize=\footnotesize,
linenos,
breaklines
]{prolog}{../src/main.pro}
