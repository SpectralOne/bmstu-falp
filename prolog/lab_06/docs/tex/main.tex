\textbf{Задание:} Используя хвостовую рекурсию, разработать программу, позволяющую найти:
\begin{enumerate}
	\item $n!$;
	\item $n$-e число Фибоначчи.
\end{enumerate}

\inputminted[
frame=single,
framesep=2mm,
baselinestretch=1.2,
bgcolor=white,
fontsize=\footnotesize,
linenos,
breaklines
]{prolog}{../src/main.pro}

\clearpage

\section*{Теоретические вопросы}

\subsection*{1. Что такое рекурсия? Как организуется хвостовая рекурсия в Prolog? Как организовать
выход из рекурсии в Prolog?}

\subsection*{2. Какое первое состояние резольвенты?}

Стек, который содержит конъюнкцию целей, истинность которых система должна доказать, называется резольвентой. Первое состояние резольвенты -- вопрос.

\subsection*{3. В каком случае система запускает алгоритм унификации? Каково назначение	использования алгоритма унификации? Каков результат работы алгоритма унификации?}

Унификация – необходима для того, чтобы определить дальнейший путь поиска решений. Унификация заканчивается конкретизацией части переменных. Алгоритм унификации -- основной шаг с помощью которого система отвечает на вопросы унификации. На вход алгоритм принимает два терма, возвращает флаг успешности унификации, и если успешно, то подстановку.

\subsection*{4. В каких пределах программы уникальны переменные?}

Областью действия переменной в Прологе является одно предложение. В разных предложениях может использоваться одно имя переменной для обозначения разных объектов.

\subsection*{5. Как применяется подстановка, полученная с помощью алгоритма унификации?}

Пусть дан терм: $(X_1, X_2,\dots, X_n)$, подстановка -- множество пар, вида: ${X_i = t_i}$, где $X_i$ -- переменная, а $t_i$ -- терм. В ходе выполнения программы выполняется связывание переменных с различными объектами, этот процесс называется конкретизацией. Это относится только к именованным переменным. Анонимные переменные не могут быть связаны со значением.

\subsection*{6. Как изменяется резольвента?}

Резольвента меняется в два этапа:

\begin{enumerate}
	\item редукция -- замена подцели телом того правила, с заголовком которого успешно унифицируется данная подцель;
	
	\item применение ко всей резольвенте подстановки.
\end{enumerate}

Резольвента уменьшается, если удаётся унифицировать подцель с фактом. Система отвечает <<Да>>, только когда резольвента становится пустой.

\subsection*{7. В каких случаях запускается механизм отката?}

Механизм отката, который осуществляет откат программы к той точке, в которой выбирался унифицирующийся с последней подцелью дизъюнкт. Для этого точка, где выбирался один из возможных унифицируемых с подцелью дизъюнктов, запоминается в специальном стеке, для последующего возврата к ней и выбора альтернативы в случае неудачи. При откате все переменные, которые были означены в результате унификации после этой точки, опять становятся свободными.

