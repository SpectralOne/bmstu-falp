\textbf{Задание:} используя хвостовую рекурсию, разработать, комментируя аргументы, эффективную программу, позволяющую:
\begin{enumerate}
	\item сформировать список из элементов числового списка, больших заданного значения;
	\item сформировать список из элементов, стоящих на нечетных позициях исходного списка (нумерация от 0):
	\item удалить заданный элемент из списка (один или все вхождения);
	\item преобразовать список в множество (можно использовать ранее разработанные процедуры).
\end{enumerate}

\clearpage
\inputminted[
frame=single,
framesep=2mm,
baselinestretch=1.2,
bgcolor=white,
fontsize=\footnotesize,
linenos,
breaklines
]{prolog}{../src/main.pro}

\clearpage

\section*{Теоретические вопросы}

\subsection*{1. Как организуется хвостовая рекурсия в Prolog?}

\subsection*{2. Какое первое состояние резольвенты?}

Стек, который содержит конъюнкцию целей, истинность которых система должна доказать, называется резольвентой. Первое состояние резольвенты -- вопрос.

\subsection*{3. Каким способом можно разделить список на части, какие, требования к частям?}

\subsection*{4. Как выделить за один шаг первые два подряд идущих элемента списка? Как выделить 1-й и 3-й элемент за один шаг?}

\subsection*{5. Как формируется новое состояние резольвенты?}

Резольвента меняется в два этапа:

\begin{enumerate}
	\item редукция -- замена подцели телом того правила, с заголовком которого успешно унифицируется данная подцель;
	
	\item применение ко всей резольвенте подстановки.
\end{enumerate}

Резольвента уменьшается, если удаётся унифицировать подцель с фактом. Система отвечает <<Да>>, только когда резольвента становится пустой.

\subsection*{6. Когда останавливается работа системы? Как это определяется на формальном уровне?}
