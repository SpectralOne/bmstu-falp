\section*{Задание 1}

Используя только функции {\texttt{car}} и {\texttt{cdr}}, написать выражения, возвращающие второй, третий, четвёртый элементы заданного списка:

\begin{lstinputlisting}[
	caption={Задание 1},
	label={lst:t1},
	style={lsp},
	linerange={3-7},
	]{../src/main.lsp}
\end{lstinputlisting}

\section*{Задание 2}

Что будет в результате выполнения выражений?

\begin{lstinputlisting}[
	caption={Задание 2},
	label={lst:t2},
	style={lsp},
	linerange={11-14},
	]{../src/main.lsp}
\end{lstinputlisting}

\section*{Задание 3}

Напишите результат вычисления выражений:

\begin{lstinputlisting}[
	caption={Задание 3},
	label={lst:t3},
	style={lsp},
	linerange={16-31},
	]{../src/main.lsp}
\end{lstinputlisting}

\section*{Задание 4}

Написать:
\begin{itemize}
	\item функцию {\texttt{(f ar1 ar2 ar3 ar4)}}, возвращающую список {\texttt{((ar1 ar2) (ar3 ar4))}};
	\item функцию {\texttt{(f ar1 ar2)}}, возвращающую список {\texttt{((ar1) (ar2))}};
	\item {\texttt{(f ar1)}}, возвращающую список {\texttt{(((ar1)))}}.
\end{itemize}

результаты представить в виде списочных ячеек.

\begin{lstinputlisting}[
	caption={Задание 4},
	label={lst:t4},
	style={lsp},
	linerange={34-41},
	]{../src/main.lsp}
\end{lstinputlisting}

Представление полученных списков в виде списочных ячеек оформлено на тетрадном листе.

\section*{Контрольные вопросы}

\subsection*{1. Классификация функций языка {\texttt{Lisp}}}

Функции в языке {\texttt{Lisp}}:
\begin{itemize}
	\item чистые (с фиксированным количеством аргументов) -- математические функции;
	\item рекурсивные функции;
	\item специальные функции -- формы (принимают произвольное количество аргументов или по разному обрабатывают аргументы);
	\item псевдофункции (создающие <<эффект>> - отображающие на экране процесс обработки данных и т. п.);
	\item функции с вариативными значениями, выбирающие одно значение;
	\item функции высших порядков -- функционалы (используются для построения синтаксически управляемых программ).
\end{itemize}

\subsection*{2. Базис языка {\texttt{Lisp}}}

Базис языка представлен:
\begin{itemize}
	\item структурами, атомами;
	\item функциями:\\
	{\texttt{atom, eq, cons, car, cdr,}}\\
	{\texttt{cond, quote, lambda, eval, label}}.
\end{itemize}

\subsection*{3. Функции {\texttt{car, cdr}}}

Являются базовыми функциями доступа к данным. {\texttt{car}} принимает точечную пару или список в качестве аргумента и возвращает первый элемент или {\texttt{Nil}}, {\texttt{cdr}} -- возвращает все элементы, кроме первого или {\texttt{Nil}}.

\subsection*{4. Функции {\texttt{list, cons}}}

Являются функциями создания списков ({\texttt{cons}} -- базовая, {\texttt{list}} -- нет). {\texttt{cons}} создаёт списочную ячейку и устанавливает два указателя на аргументы. {\texttt{list}} принимает переменное число аргументов и возвращает список, элементами которого являются аргументы, переданные в функцию.