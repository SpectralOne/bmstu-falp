\chapter*{Лабораторная работа №1}

\section*{Задание 1}

Представить следующие списки в виде списочных ячеек:
\begin{enumerate}
	\item {\texttt{'(open close halp)}}
	\item {\texttt{'((open1) (close2) (halp3))}}
	\item {\texttt{'((one) for all (and (me (for you))))}}
	\item {\texttt{'((TOOL) (call))}}
	\item {\texttt{'((TOOL1) ((call2)) ((sell)))}}
	\item {\texttt{'(((TOOL) (call)) ((sell)))}}
\end{enumerate}

Решение оформлено на тетрадном листе.

\section*{Контрольные вопросы}

\subsection*{1. Перечислить элементы языка {\texttt{Lisp}}}

Элементами языка {\texttt{Lisp}} являются атомы и точечные пары (структуры). К атомам относятся:
\begin{itemize}
	\item символы (идентификаторы) -- набор литер, начинающихся с буквы;
	\item специальные символы для обозначения логических констант {\texttt{T,~Nil}};
	\item самоопределимые атомы -- натуральные числа, дробные числа, вещественные числа, строки (последовательность символов, заключённых в двойные апострофы).
\end{itemize}

\subsection*{2. Синтаксис элементов языка и их представление в памяти}

\noindent{\texttt{Точечные пары ::= (<атом>, <атом>) |}}\\
{\texttt{(<атом>, <точечная пара>) |}}\\
{\texttt{(<точечная пара>, <атом>) |}}\\
{\texttt{(<точечная пара>, <точечная пара>)}}\\

\noindent{\texttt{Список ::= <пустой список> | <непустой список>)}}, где\\
{\texttt{<пустой список> ::= () | Nil}},\\
{\texttt{<непустой список> ::= (<первый элемент>, <хвост>) }},\\
{\texttt{<первый элемент> ::= (S-выражение)}},\\
{\texttt{<хвост> ::= <список>}}\\

\noindent Список -- частный случай {\texttt{S-выражения}}.\\
Синтаксически любая структура (точечная пара или список) заключается в {\texttt{()}}:\\
{\texttt{(A . B)}} -- точечная пара\\
{\texttt{(A)}} -- список из одного элемента\\
Пустой список изображается как {\texttt{Nil}} или {\texttt{()}}\\
Непустой список может быть изображён: {\texttt{(A. (B . (C ())))}} или {\texttt{(A B C)}}\\
Элементы списка могут являться списками: {\texttt{((A) (B) (C))}}\\
Любая непустая структура {\texttt{Lisp}} в памяти представлена списковой ячейкой, хранящей два указателя: на голову (первый элемент) и хвост (всё остальное).

\subsection*{3. Как воспринимается символ {\texttt{'}}}

Символ {\texttt{'}} -- функциональная блокировка, эквивалентен функции {\texttt{quote}}. Блокирует вычисление выражения. Таким образом, выражение воспринимается интерпретатором как данные.

\subsection*{4. Что такое рекурсия и примеры рекурсии из языка {\texttt{Lisp}}}

Рекурсия -- ссылка на описываемый объект в процессе его описания. Списки в {\texttt{Lisp}} заданы рекурсивно, т.е. каждый элемент списка является ещё одним списком, имеющим пустой или непустой хвост.