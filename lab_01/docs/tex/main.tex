\section*{Задание 1}

Представить следующие списки в виде списочных ячеек:
\begin{enumerate}[topsep=0pt]
	\item {\texttt{'(open close halp)}}
	\item {\texttt{'((open1) (close2) (halp3))}}
	\item {\texttt{'((one) for all (and (me (for you))))}}
	\item {\texttt{'((TOOL) (call))}}
	\item {\texttt{'((TOOL1) ((call2)) ((sell)))}}
	\item {\texttt{'(((TOOL) (call)) ((sell)))}}
\end{enumerate}

Решение оформлено на тетрадном листе.

\section*{Задание 2}


Используя только функции {\texttt{car}} и {\texttt{cdr}}, написать выражения, возвращающие второй, третий, четвёртый элементы заданного списка:


\begin{lstinputlisting}[
	caption={Задание 1},
	label={lst:t1},
	style={lsp},
	linerange={3-7},
	]{../src/main.lsp}
\end{lstinputlisting}

\section*{Задание 3}


Что будет в результате выполнения выражений?


\begin{lstinputlisting}[
	caption={Задание 2},
	label={lst:t2},
	style={lsp},
	linerange={11-14},
	]{../src/main.lsp}
\end{lstinputlisting}

\section*{Задание 4}


Напишите результат вычисления выражений:


\begin{lstinputlisting}[
	caption={Задание 3},
	label={lst:t3},
	style={lsp},
	linerange={16-31},
	]{../src/main.lsp}
\end{lstinputlisting}

\section*{Задание 5}


Написать:

\begin{itemize}[topsep=0pt, noitemsep]
	\item функцию {\texttt{(f ar1 ar2 ar3 ar4)}}, возвращающую список {\texttt{((ar1 ar2) (ar3 ar4))}};

	\item функцию {\texttt{(f ar1 ar2)}}, возвращающую список {\texttt{((ar1) (ar2))}};

	\item {\texttt{(f ar1)}}, возвращающую список {\texttt{(((ar1)))}}.

\end{itemize}


результаты представить в виде списочных ячеек.


\begin{lstinputlisting}[
	caption={Задание 4},
	label={lst:t4},
	style={lsp},
	linerange={34-41},
	]{../src/main.lsp}
\end{lstinputlisting}

Представление полученных списков в виде списочных ячеек оформлено на тетрадном листе.



\section*{Контрольные вопросы}

\subsection*{1. Элементы языка: определение, синтаксис, представление в памяти}

\subsubsection{Элементы языка}

Элементами языка {\texttt{Lisp}} являются атомы и точечные пары (структуры). К атомам относятся:
\begin{itemize}[topsep=0pt, noitemsep]
	\item символы (идентификаторы) -- набор литер, начинающихся с буквы;
	\item специальные символы для обозначения логических констант {\texttt{T,~Nil}};
	\item самоопределимые атомы -- натуральные числа, дробные числа, вещественные числа, строки (последовательность символов, заключённых в двойные апострофы).
\end{itemize}

\subsubsection*{Синтаксис элементов языка и их представление в памяти}

\noindent{\texttt{Точечные пары ::= (<атом>, <атом>) |}}\\
{\texttt{(<атом>, <точечная пара>) |}}\\
{\texttt{(<точечная пара>, <атом>) |}}\\
{\texttt{(<точечная пара>, <точечная пара>)}}\\

\noindent{\texttt{Список ::= <пустой список> | <непустой список>)}}, где\\
{\texttt{<пустой список> ::= () | Nil}},\\
{\texttt{<непустой список> ::= (<первый элемент>, <хвост>) }},\\
{\texttt{<первый элемент> ::= (S-выражение)}},\\
{\texttt{<хвост> ::= <список>}}\\

\noindent Список -- частный случай {\texttt{S-выражения}}.\\
Синтаксически любая структура (точечная пара или список) заключается в {\texttt{()}}:\\
{\texttt{(A . B)}} -- точечная пара\\
{\texttt{(A)}} -- список из одного элемента\\
Пустой список изображается как {\texttt{Nil}} или {\texttt{()}}\\
Непустой список может быть изображён: {\texttt{(A. (B . (C ())))}} или {\texttt{(A B C)}}\\
Элементы списка могут являться списками: {\texttt{((A) (B) (C))}}\\
Любая непустая структура {\texttt{Lisp}} в памяти представлена списковой ячейкой, хранящей два указателя: на голову (первый элемент) и хвост (всё остальное).

\subsection*{2. Особенности языка \texttt{LISP}. Структура программы. Символ апостроф}

\subsubsection*{Структура программы}

Lisp - язык символьной обработки. 
В Lisp программа и данные представлены списками.
По умолчанию список считается вычислимой формой, в которой 1 элемент - название функции, остальные элементы - аргументы функции.

\subsubsection*{Особенности языка}

Т.к. и программа и данные представлены списками, то их нужно как-то различать. 
Для этого была создана функция quote, а ' - ее сокращенное обозначение. 
quote - функция, блокирующая вычисление.

\subsubsection*{Символ апостроф}

Символ {\texttt{'}} -- функциональная блокировка, эквивалентен функции {\texttt{quote}}. Блокирует вычисление выражения. Таким образом, выражение воспринимается интерпретатором как данные.

\subsection*{3. Базис языка \texttt{LISP}. Ядро языка}

Базис языка представлен:

\begin{itemize}[topsep=0pt, noitemsep]
	\item структурами, атомами;

	\item функциями:\\

	{\texttt{atom, eq, cons, car, cdr,}}\\

	{\texttt{cond, quote, lambda, eval, label}}.

\end{itemize}